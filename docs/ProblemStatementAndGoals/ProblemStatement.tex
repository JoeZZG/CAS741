\documentclass{article}

\usepackage{tabularx}
\usepackage{booktabs}
\usepackage{hyperref}

\title{Problem Statement and Goals\\\progname}

\author{Zhuo Zhang}

\date{2026-01-16}

\input{../Comments.text}
\input{../Common.text}

\begin{document}

\maketitle

\begin{table}[hp]
\caption{Revision History} \label{TblRevisionHistory}
\begin{tabularx}{\textwidth}{llX}
\toprule
\textbf{Date} & \textbf{Developer(s)} & \textbf{Change}\\
\midrule
2026-01-16 & Zhuo Zhang & Initial draft\\
\bottomrule
\end{tabularx}
\end{table}

\section{Problem Statement}

\subsection{Problem}

In electromagnetism, charged particles undergo \href{https://cmms.triumf.ca/~jess/lab/9/node2.html}{deflection} (a displacement results from the Lorentz Force), circular motion and
\href{https://physics.bu.edu/~duffy/Ejs/EP_chapter19/Velocity_Selector_v1d_Intro\%201.html}{velocity selection} under electric and magnetic fields. Analyzing these behaviors
can help understanding phenomena such as velocity selectors, Cathode Ray
Tube deflection and other basic principles of mass separation. The problem is to
provide a computational tool that predicts trajectories and deflection of charged
particle when it passes through uniform electric or magnetic field regions under different configurations. The importance
of this problem is from that the motion is described by a system of ordinary
differential equations (ODEs).


\subsection{Inputs and Outputs}

The inputs are particle properties, field specification, geometry of the experiment, and termination condition of the experiment. The outputs are representation of particles' trajectory and deflection.

\subsection{Stakeholders}
\begin{itemize}
\item Primary use stakeholders: instructors or students who want to confirm or analyze motions of charged-particle in different conditions of fields.
\item Project stakeholders: the Drasil maintainers and course instructors who will check the artifacts, traceability and quality of the document.
\item Secondary stakeholders: future CAS741 students who wish to reuse, refer or extend the case study.
\end{itemize}

\subsection{Environment}

This software runs on personal computer with Ubuntu 22.04 LTS. User interacts via command line. Inputs are provided through a text-based configuration and outputs are given by portable and standard formats in CSV. The documentation and generated artifacts are under Drasil workflow.

\section{Goals}
\begin{itemize}
\item This software will compute motion of charged-particle by modeling dynamics. 
\item This software will support electric-field-only, magnetic-field-only, and crossed-field configuration regions.
\item This software will produce impact point, flight time, trajectory and deflection in a readable format.
\end{itemize}

\section{Stretch Goals}
\begin{itemize}
\item This software will solve for unknown parameter given a target detector location.
\item This software will add error estimation to improve accuracy across different parameters.
\end{itemize}

\end{document}
